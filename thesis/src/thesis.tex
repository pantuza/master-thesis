%
% Class document definition
%
\documentclass[msc]{template/ppgccufmg}  % [phd] | [msc]



%
% Document used packages
%
\usepackage[brazil]{babel}
\usepackage[utf8]{inputenc}
\usepackage[T1]{fontenc}
\usepackage{type1ec}
\usepackage{graphicx}
\usepackage[a4paper,
      portuguese,
      bookmarks=true,
      bookmarksnumbered=true,
      linktocpage,
      colorlinks,
      citecolor=black,
      urlcolor=black,
      linkcolor=black,
      filecolor=black,
]{hyperref}
\usepackage[square]{natbib}
\usepackage{multirow}
\usepackage[usenames,dvipsnames,table,xcdraw]{xcolor} 
\usepackage{pbox}
\usepackage{multicol}
\usepackage{hyperref}
\usepackage{listings}

\hyphenation{re-qui-si-to}

\sloppy



%
% The document body and structure
%
\begin{document}



% Definitions for ppgccufmg class
\ppgccufmg{
    title={Grafos como uma primitiva do plano de controle para análise e 
        gerenciamento de Redes Definidas por Software},
    authorrev={Pantuza, Gustavo},
    cutter={M1234x}, % INFORMAÇÃO QUE VAI NA FICHA CATALOGRÁFICA
    cdu={100.0*01.10},  % Define o identificador CDU do documento, fornecido pela Secretaria do Curso.
    university={Universidade Federal de Minas Gerais},
    course={Ciência da Computação},
    address={Belo Horizonte},
    date={2015-03},
    keywords={Redes definidas por software, Openflow, Grafos, 
        Gerenciamento de redes, Sistemas distribuídos, Redes de computadores},
    advisor={Luiz Filipe Menezes Vieira},
    abstract=[brazil]{Resumo}{src/abstract_pt},
    abstract=[english]{Abstract}{src/abstract_en},
    %abstract=[brazil]{Resumo Estendido}{resumoest}, %resumoest.tex
    %dedication={dedicatoria},
    %ack={agradecimentos},
    %  ack=[Acknowledgments]{ack},
    epigraphtext={Live long and prosper!}{Mr. Spock},
}




% Chapters and files that compose the document content
%
% Introduction
%
\section{Introdução}


%
% Community emvolvment
%
\begin{frame}\frametitle{Apresentação}

	\begin{figure}[h]
        \centering
        \includegraphics[scale=0.5]{images/community.png}
    \end{figure}
\end{frame}


%
% Motivation
%
\begin{frame}\frametitle{Motivação}
   
    \begin{itemize}
        \setlength{\itemsep}{1cm}
        \item A Internet demanda que a infraestrutura evolua em paralelo com 
            as aplicações e serviços
        \item Algoritmos em grafos são base para diversas aplicações em rede
        \item Computação feita em diferentes nós da rede repetidamente
        \item Logicamente centralizado, o plano de controle permite 
            minimizar a quantidade de computações
    \end{itemize}
\end{frame}


%
% Problem
%
\begin{frame}\frametitle{Problema}
    \begin{itemize}
        \setlength{\itemsep}{1cm}
        \item Uma visão topológica global é um dos principais aspectos do 
              paradigma das Redes definidas por software.
        \item Grafos representam de maneira natural e precisa a topologia 
            de uma rede.
        \item Grafos deveriam ser um recurso básico, uma premissa em 
            controladores SDN
    \end{itemize} 
\end{frame}


%
% Scientific contributions
%
\begin{frame}\frametitle{Contribuições científicas}
    \begin{itemize}
        \setlength{\itemsep}{1cm}
        \item Uma abstração da rede na forma de um grafo dinamicamente 
            atualizado.
        \item Avaliações do controlador, da rede e do protocolo OpenFlow
        \item Avaliação de grafos como primitiva em SDN
    \end{itemize}
\end{frame}


\section{Fundamentação teórica}

\section{Redes definidas por software (SDN)}


\subsection{Definição}


\subsection{Plano de dados}


\subsection{Plano de controle}


\subsection{Características}

\section{Openflow}



\subsection{Definição}

Em 2008 o protocolo OpenFlow foi publicado. Ele permitiu que pesquisadores
pudessem criar experimentos com novos protocolos em redes convencionais
\citep{nick2008openflow}.
O OpenFlow foi criado como um padrão aberto, o que permite que todos os 
fabricantes de equipamentos de redes possam habilitar seus produtos ao 
padrão.

O protocolo consiste em uma interface de programação para o switch. 
Assim, um programador pode, através de um programa, controlar a forma como 
um switch executa seu encaminhamento de pacotes. 
De uma maneira bem clara, o protocolo OpenFlow separa o plano de dados
do plano de controle, fazendo com que soluções SDN possam ser criadas 
e experimentadas.
Por ser uma solução de baixo custo, o OpenFlow obteve boa aceitação na 
academia e no mercado, dado o volume de empresas e pesquisas relacionadas
ou que utilizam o protocolo.

\subsection{Componentes}

Na arquitetura estabelicida pelo protocolo OpenFlow existem dois papéis
principais.
O controlador e o switch OpenFlow. 
Uma separação baseada no modelo das Redes definidas por software (SDN)
conforme pode ser visto na figura~\ref{fig:of-arch}.

\begin{figure}[h!]
    \centering
    \label{fig:of-arch}
    \includegraphics{img/openflow-architecture}
    \caption{Divisão da arquitetura OpenFlow}
\end{figure}

\subsection{Arquitetura do Switch}

O switch OpenFlow pode ser dividido em três principais partes. 
A primeira é o canal de comunicação seguro com o controlador.
A segunda é a interface do protocolo OpenFlow que permite ao controlador
controlar o switch.
A terceira é sua tabela de fluxos.

O canal de comunicação seguro tem a garantia de confiabilidade na troca 
de mensagens entre o controlador e o switch através do protocolo SSL 
(Secure Socket Layer). 
Isso adiciona proteção à rede em relação a ataques de elementos mal 
intencionados \citep{rothenberg2010openflow}.

A interface do protocolo OpenFlow é a padronização das mensagens enviadas 
pelo controlador ao switch de modo a definir o comportamento do encaminhamento
de pacotes. 
Ou seja, é o conjunto de possíveis instruções para se controlar de maneira
genérica o plano de dados de qualquer \emph{hardware} de redes habilitado 
ao padrão OpenFlow.

A tabela de fluxos é composta por regras.
Cada regra consiste em ações associadas à fluxos. 
Através dessa tabela o switch executa o encaminhamento de pacotes. 
As entradas dessa tabela são atualizadas pelo controlador.

\begin{figure}[h!]
    \centering
    \label{fig:switch-arch}
    \includegraphics[width=\linewidth]{img/switch-architecture}
    \caption{Arquitetura do switch OpenFlow}
\end{figure}


\subsection{Fluxos}

Um fluxo é a representação de um ou mais pacotes em função de suas
características.
Essas características variam de acordo com o cabeçalho OpenFlow.
Os fluxos são genéricos. 
Ao invés de analisar pacotes e portas para fazer a análise do trafego,
que representam o passado, o fluxos são a representação do futuro. 
Um fluxo pode caracterizar pacotes que ainda não passaram pelo switch mas 
que possuem as mesmas características que compõem aquele fluxo. 
Sendo assim, a ação associada ao fluxo é aplicada a esses novos pacotes.

A tabela de fluxos dentro do switch OpenFlow identifica os fluxos para que o 
plano de dados execute ações sobre os pacotes que são pertencentes àquele
fluxo. 
A tabela \ref{tbl:flowtable} apresenta de maneira simplificada a tabela 
de fluxos dentro do switch OpenFlow.

\begin{table}
    \centering
    \begin{tabular}{ | l | l | l | l |}
    \hline
    \textbf{Header} & \textbf{Counters} & \textbf{Actions} & \textbf{Priority} 
    \\ \hline in\_port=5 & 55635 bytes & \pbox{20cm}{Forward \\ port=8} 
    & 100 \\ \hline
    \pbox{20cm}{ip=192.168.1.42 \\ port=80} & 4032 bytes & \pbox{30cm}{Set 
    \\ rewrite \\ ip=192.168.1.100} & 500 \\ \hline ipproto=UDP & 100 bytes 
    & Drop & 700 \\ \hline
    \end{tabular}
    \caption{Tabela de fluxos simplificada}
    \label{tbl:flowtable}
\end{table}

\subsection{Cabeçalho}

O cabeçalho OpenFlow se estende da camada 1 até a camada 4 da pilha TCP/IP.
A figura \ref{fig:of-header} apresenta os campos e suas respectivas camadas.

\begin{figure}[h!]
    \centering
    \label{fig:of-header}
    \includegraphics[width=\linewidth]{img/openflow-header}
    \caption{Cabeçalho OpenFlow}
\end{figure}

Quando um novo pacote ingressa no switch OpenFlow esse cabeçalho é preenchido
e encaminhado para o controlador. 
Através da análise dessas informações do fluxo o controlador envia uma 
mensagem ao switch instalando/atualizando regras na tabela de fluxos. 

Esse mecanismo de rotulagem e identificação de tráfego e pacotes (fluxos) é
inspirado no MPLS (\emph{Multi-protocol Label Switching}) 
\citep{bruce2008mpls}.

\subsection{Ações}

As ações (\emph{actions}) são aplicadas aos fluxos especificados na tabela
de fluxos do switch.
Ações como encaminhamento, atualização de cabeçalho e rejeição podem ser 
aplicadas aos pacotes pertencentes àquele fluxo.

A especificação do protocolo OpenFlow \citep{ofprotocol2015} descreve todas 
as possíveis ações que podem ser aplicadas.
A lista abaixo apresenta algumas ações:

\begin{multicols}{3}
    \begin{itemize}
        \item Forwarding
        \item Drop
        \item Set
        \item Strip
        \item Copy-in
        \item Copy-out
        \item Push
        \item Pop
        \item Dec
    \end{itemize}
\end{multicols}

\subsection{Controlador}

\section{architecture}


%
% Archtecture
%
\begin{frame}\frametitle{Arquitetura Openflow}

    \begin{itemize}
    \item Temos dois papeis principais:
        \begin{itemize}
        \item Controlador
        \item Switch Openflow
        \end{itemize}
    \item Algo te lembra plano de dados e controle desacoplados?
    \end{itemize}
    
	\begin{figure}[h]
        \centering
        \includegraphics[scale=0.6]{images/controller-to-openflow.png}
    \end{figure}
\end{frame}



%
% Switch Archtecture
%
\begin{frame}\frametitle{Arquitetura do Switch Openflow}

    \begin{itemize}
    \item Internamente um Switch openflow é assim:
    \end{itemize}
    
	\begin{figure}[h]
        \centering
        \includegraphics[scale=0.5]{images/openflow-switch-architecture.png}
    \end{figure}
\end{frame}

%
% Switch Archtecture
%
\begin{frame}\frametitle{Arquitetura Openflow}

    \begin{itemize}
    \item \textbf{Openflow secure channel}: É a conexão segura entre o
          controlador e o switch openflow
    \vspace*{0.5cm}
    \item \textbf{Flow table}: É a tabela onde são identificados os fluxos
    \vspace*{0.5cm}
    \item Para cada fluxo tem-se uma ação (action) a ser tomada
    \end{itemize}
\end{frame}

%
% Switch Archtecture
%
\begin{frame}\frametitle{Arquitetura Openflow}

    \begin{itemize}
    \item Um fluxo é identificado pelos seguintes campos do cabeçalho 
          Openflow:
    \end{itemize}
	\begin{figure}[h]\hspace*{-1.2cm}
        \centering
        \includegraphics[scale=0.32]{images/openflow-header.jpg}
    \end{figure}
    
\end{frame}


%
% Actions
%
\begin{frame}\frametitle{Actions}

	\begin{figure}[h]
        \centering
        \includegraphics[scale=0.5]{images/action.png}
    \end{figure}
    
\end{frame}


%
% Actions
%
\begin{frame}\frametitle{Tipos de Actions}

    \begin{itemize}
    \item Forwarding
    \item Drop
    \item Set
    \item strip
    \item Copy-in
    \item Copy-out
    \item Push
    \item Pop
    \item Dec
    \end{itemize}

\end{frame}


%
% Flow table 
%
\begin{frame}\frametitle{Tabela de Fluxos}

\begin{center}
    \begin{tabular}{ | l | l | l | l |}
    \hline
    \textbf{Header} & \textbf{Counters} & \textbf{Actions} & \textbf{Priority} \\ \hline
    in\_port=5 & 55635 bytes & \pbox{20cm}{Forward \\ port=8} & 100 \\ \hline
    \pbox{20cm}{ip=192.168.1.42 \\ port=80} & 4032 bytes & \pbox{30cm}{Set \\ rewrite \\ ip=192.168.1.100} & 500 \\ \hline
    ipproto=UDP & 100 bytes & Drop & 700 \\ \hline
    \end{tabular}
\end{center}

\end{frame}

%
% Controller
%
\begin{frame}\frametitle{Controlador}

    \begin{itemize}
    \item É um software que se conecta de maneira segura ao switch openflow
          com o objetivo de manipular sua tabela de fluxos
    \item Esse software pode ser distribuído
    \item Ele representa uma entidade lógica e centralizada
    \item É possível ter visão e controle de estado global da rede
    \item Permite que outros serviços e programas façam requisições e troca
          de mensagem com o plano de controle da rede
    \item Pode calcular estatísticas da rede
    
    \end{itemize}

\end{frame}


%
% Controller
%
\begin{frame}\frametitle{Controlador}

    \begin{itemize}
    \item Cabe ao programador lidar com os problemas típicos em 
          desenvolvimento de software:
          \begin{itemize}
          \item Tolerância a falha
          \item Persistência
          \item Eficiência
          \item design de implementação
          \item debugging
          \item Testes
          \end{itemize}
    \end{itemize}
\end{frame}

%
% Controller
%
\begin{frame}\frametitle{Controlador}

    \begin{itemize}
    \item Controladores Openflow:
          \begin{itemize}
          \item Biblioteca \href{http://opennetworkingfoundation.github.io/libfluid/index.html}{Libfluid}
                para criação de aplicações/controladores em SDN
          \item \href{http://www.noxrepo.org/nox/about-nox/}{Nox Controller}
          \item \href{https://openflow.stanford.edu/display/Beacon/Home}{Beacon}
          \item \href{http://www.noxrepo.org/pox/about-pox/}{Pox Controller}
          \item \href{http://osrg.github.io/ryu/}{Ryu}
          \end{itemize}
    \end{itemize}
\end{frame}


%
% Simple topology
%
\begin{frame}\frametitle{Arquitetura Openflow}

    \begin{itemize}
    \item Uma topologia simples:
    \end{itemize}
    
	\begin{figure}[h]
        \centering
        \includegraphics[scale=0.3]{images/simple-topology.png}
    \end{figure}
\end{frame}


%
% N openflow 
%
\begin{frame}\frametitle{Arquitetura Openflow}

    \begin{itemize}
    \item Um controlador para vários \emph{Switches}
    \end{itemize}
    
	\begin{figure}[h]
        \centering
        \includegraphics[scale=0.4]{images/n-openflow-switches.png}
    \end{figure}
\end{frame}


%
% SDN inter domain
%
\begin{frame}\frametitle{Arquitetura Openflow}

    \begin{itemize}
    \item Comunicação entre domínios de rede
    \end{itemize}
    
   
	\begin{figure}[h]\hspace*{-1cm}
        \centering
        \includegraphics[scale=0.3]{images/edge-core-sdn.png}
    \end{figure}
\end{frame}



%
% Distributed openflow controller
%
\begin{frame}\frametitle{Arquitetura Openflow}

    \begin{itemize}
    \item Controlador distribuído
    \end{itemize}
    
	\begin{figure}[h]
        \centering
        \includegraphics[scale=0.4]{images/distributed_sdn_controller.png}
    \end{figure}
\end{frame}

\chapter{Trabalhos relacionados}

Essa seção apresenta os trabalhos relacionados ao presente projeto de 
dissertação e discute suas características.
A seção segue uma linha cronológica com temas e artigos que levam a 
formulação que é a base do presente trabalho.

\section{As redes em camadas (\emph{Overlay})}

Um entendimento das implicações de redes \emph{overlays} para a 
arquitetura da Internet, para o mercado e para a política
é apresentado em \citep{clark2006overlay}. 
De modo geral o artigo descreve, principalmente, as redes de 
CDN (Redes de entrega de conteúdo), segurança e roteamento em camadas. 
O artigo posiciona as \emph{overlays} como uma uma camada intermediária, 
acima dos protocolos básicos IP e abaixo da camada de 
aplicação. 
Segundo essa visão, \emph{overlays} são para a internet básica (ponto-a-ponto)
usuários finais em que, por exemplo, um roteador apenas encaminha pacotes sem
se importar com seu conteúdo ou finalidade. 
Por outro lado, para a aplicação ela se comporta como sua infraestrutura. 
Segundo o autor, as \emph{overlays} se tornaram o principal meio de 
evolução da arquitetura da Internet.

Utilizar \emph{overlay} para tratar as deficiências da rede é custoso, no
entanto atualizar a infraestrutura da Internet básica seria ainda mais. 
As redes \emph{overlays} irão, de maneira disruptiva, representar o modo de 
inovar dentro da Internet criando novos jogadores, 
um novo cenário econômico e novas regras.

A arquitetura RON (\emph{Resilient Overlay Network}) 
\citep{anderson2001resilient} é capaz de tornar a entrega de pacotes na 
Internet mais confiável através de detecção e recuperação de interrupções e 
falhas no roteamento. 
A Internet foi criada como uma rede \emph{overlay} que funcionava sobre a rede 
de telefonia. Ou seja, o conceito de Redes \emph{overlay} não é uma idéia nova.
No entanto, poucas dessas redes foram estruturadas para se recuperar e tolerar 
falhas de forma eficiente. Nesse contexto é que as RONs demonstram ser 
confiáveis.

As RONs possuem três Metas/objetivos. O primeiro objetivo e principal é 
permitir que um grupo de nós possam se comunicar independente de haver uma 
falha na rota entre eles. 
Isso torna o roteamente confiável. O segundo objetivo
é tornar o roteamento e a seleção de próximo host com aplicações distribuídas 
mais forte, mais eficiente do que tradicionalmente é feito com outras 
arquiteturas e protocolos. 
O terceiro e último objetivo é fornecer um framework 
para implementação de políticas de roteamento que governam a escolha de rotas 
dentro da rede. 

As implicações de desempenho entre a rede \emph{overlay} Gnutella e a 
infra-estrutura da Internet é descrito em \citep{ripeanu2002mapping}. 
Redes P2P agregam vários computadores que entram e saem da rede todo o tempo. 
Esses Peers (computadores) podem não ter um endereço IP permanente. 
As redes P2P são entidades independentes e auto organizadas. 
Esse trabalho descreve um período de avaliação de sete meses do crescimento das 
redes Gnutella assim como suas implicações de desempenho. 
A rede é composta por serventes (computadores) que são os nós da rede virtual 
na camada de aplicação em que seus enlaces são formados por conexões 
TCPs abertas. 
Ao analisar a topologia utilizada pelo Gnutella o artigo explicita que apesar 
de essas redes serem eficientes em lidar com falhas aleatórias em computadores,
elas estão vulneráveis a ataques bem planejados.

\section{Redes centradas em conteúdo}

\emph{Content-Centric Networking} (CCN, redes centradas no conteúdo) trata o 
conteúdo como uma primitiva\citep{van2009networking}. 
Esse conteúdo é requisitado através de um nome. 
Em analogia ao IP, a CCN substitui o 'onde', utilizado no IP, pelo 'o que'. 
A comunicação dentro da CCN é definida por 'consumidores de dados'. 
Existem dois tipos de pacotes: \emph{Interest} (interesse) 
e \emph{Data} (dado).    

O documento descreve o funcionamento completo do roteamento dos pacotes 
baseados em conteúdo, assim como seria seu comportamento dentro de 
intra-domínios e inter-domínios. 
A segurança acontece no nível do dado ao invés de apenas uma propriedade da 
conexão pela qual o dado trafega. 
O projeto de CCN's a protege de várias classes de ataques de rede. 
O artigo apresenta comparações e avaliações de CCN's em relação ao TCP/IP. 
São relacionados temas como eficiência na tranferência de dados, 
eficiência na distribuição de conteúdo, a estratégia de camadas da rede 
e voz sobre CCN. 
A CCN foi projetada para substituir o IP, mas pode ser distribuída 
como um overlay. 


\section{Computação em arquiteturas na nuvem (\emph{cloud})}

\section{O protocolo OpenFlow e as inovações em rede}

\section{Recuperação de informação topológica}

A necessidade de se buscar e monitorar dados 
específicos, principalmente em redes complexas, atrai o desenvolvimento
de linguagens DSL (linguagens de domínio especifico) que simplifique, 
organize, generalize e garanta eficiência na manipulação desses dados.
Esse é o exemplo do Frenetic \citep{foster2011frenetic} 
e do Pyretic \citep{monsanto2013composing}.

Enquanto a abordagem da DSL é baseada em atuação, execução,
conjunta com o controlador,
o NIB é baseada em um banco de dados distribuído,
com uma abordagem mais ampla e atacando problemas mais gerais
como persistência, concorrência, redundância, escalabilidade, etc.

\section{A abordagem em grafos}

A abordagem de representar a rede na forma de um grafo foi mencionada 
por Casado \emph{et al.} em um dos primeiros artigos sobre SDN
\citep{martin2010virtualizing}.
No entanto, nenhum detalhe de implementação é apresentado.
Em um trabalho futuro, uma solução SDN foi desenvolvida através de 
diferentes topologias de rede dentro do contexto de \emph{datacenter} 
em que a abstração em grafos não foi adotada \citep{ripcord}. 

Raghavendra \emph{et al.} apresenta um módulo em grafos com capacidade 
de atualização dinâmica com uma API para algorítmos em grafos
\citep{ramya2012dynamic}.
Esse trabalho não possui nenhuma integração com algum controlador SDN,
que é a base da avaliação do presente trabalho.

O controlador \emph{Onix} \citep{teemu2010onix} foi projetado em torno do 
conceito NIB (\emph{Network Information Base}), que é uma base 
de informações da rede.
Essa base mantém uma visão global da rede de maneira similar à 
MIB (\emph{Management Information Base}) implementada sobre o
protocolo SNMP.
Essa representação baseada em grafos é alcançada indexando cada
entrada de elemento em relação a seus vizinhos.


\chapter{Uma avaliação do OpenFlow}

Esse capítulo avalia o protocolo OpenFlow através de um sistema de
balanceamento de carga.
Através dos recursos fornecidos pelo protocolo é possível maximizar a justiça
no balanceamento de carga em um serviço HTTP.
Detalhes sobre a implementação e os resultados desse experimentos são 
apresentados nesse capítulo.

\section{Proposta}

Sistemas de balanceamento de carga são baseados em políticas de balanceamento.
Essas políticas podem se aproveitar positivamente da separação do plano de 
dados e do plano de controle e sua flexibilidade.
O presente trabalho apresenta um sistema de balanceamento de carga inteligente
baseado em Redes Definidas por \emph{Software} (SDN).
As políticas de balanceamento são baseadas no trafego da rede, na carga dos 
servidores e no estado corrente do serviço.
Os experimentos mostram que a abordagem em SDN permite um balancemanto de carga
que reduz a carga dos servidores, aumenta a disponibilidade do serviço e 
otimiza a instalação de fluxos no plano de dados.

\section{Introdução}

Os serviços modernos devem ser escaláveis para atender a milhões ou milhares
de clientes.
Para realizar essa tarefa, os serviços devem ser distribuídos em vários
servidores. 
Como consequência, para garantir a experiência do usuário/cliente, o volume
de consultas em processamento por cada servidor deve ser compatível com 
sua capacidade.

Balancemento de carga é um requisito para sistemas distribuídos que se 
escalam através de vários servidores.
Conforme apresentado em \citep{hardeep2010openflow}, balanceamento de carga
em Redes de computadores consiste em uma técnica usada para distribuir a 
carga de trabalho entre vários enlaces ou computadores.
Um balanceamento de carga deve ser transparente para o usuário final e 
permitir que a aplicação seja escalável e flexível.

As soluções atuais são muito caras. 
Normalmente, dispositivos intermediários (\emph{middleboxes}) são utilizados,
no entanto não são customizáveis \citep{richard2011openflow}.
Balanceadores de carga são caros e, em muitos casos, tornam-se um ponto de 
congestionamento da rede \citep{nikhil2010asterix}.
Os dispositivos intermediários para balanceamento de carga não levam em conta
a largura de banda ou a latência da rede.
Contudo, alguns balanceadores não conseguem agrupar as requisições por 
similaridade \citep{richard2011openflow}, nem avaliar diretamente a topologia
da rede como em \citep{nikhil2009plugnserve}.


Uma solução viável seria executar a tarefa de balanceamento de carga em 
comutadores comerciais.
O protocolo OpenFlow permite que, políticas de balanceamento de carga, sejam
aplicadas a rede com baixo custo e que podem se adaptar à aplicação ou serviço.

A solução proposta nessa avaliação, é aproveitar da flexibilidade do plano de
controle separado do plano de dados.
Como uma entidade logicamente centralizada, o controlador possui uma visão 
topológica global da rede \citep{nick2008openflow}.
Os experimentos mostram que a abordagem em SDN a perda de pacotes do serviço
e seu atraso.
A disponibilidade e a vazão da aplicação é maximizada.
As políticas de balanceamento de carga propostos não distribuem igualmente as 
requisições entre os servidores. 
Ao invés disso, elas garantem que a carga de trabalho ao longo dos servidores
da aplicação sejam mais justos e homogênea.

Um sistema de balanceamento de carga corporativo utilizando OpenFlow é 
apresentado em \citep{nikhil2009plugnserve}.
Uma generalização do fluxo de pacotes através de \emph{wildcards} (fluxos
curinga) foi feita em \citep{richard2011openflow}.
Uma medição da latência e da largura de banda é descrita em 
\citep{hardeep2010openflow}.
Um modelo de protocolo para balanceamento de carga foi proposto por 
\citep{charalampos2013modelling}.
Esse trabalho apresenta uma proposta que verifica a carga de servidores de 
aplicação, a fila de requisições (HTTP), a carga prevista e o número de 
requisições respondidads.
Esses parâmetros são a basa do balanceamento de carga. 
Os fluxos de ida e de volta (fluxo reverso) são instalados nos comutadores a
fim de reduzir o número de pacotes enviados ao controlador.

Primeiramente apresentamos o projeto do sistema de balanceamento de carga.
Em seguida, são discutidas as políticas de balanceamento de carga.
Os experimentos e a avaliação dos resultados são apresentados.
Ao final, são discutidos trabalhos futuros e uma conclusão.

\section{Trabalhos relacionados}

Balancemento de carga é importante para serviços que exigem robustês ao dividir
a carga de entrega de pacotes de modo agregado.
O OpenFlow permite criar soluções de baixo custo.
Considerar o balanceamento de carga uma primitiva em Redes de computadores é 
apresentado em \cite{nikhil2010asterix}.

Medições de latência e largura de banda foram feitos através do controlador
NOX em \cite{hardeep2010openflow}.
Abordagens utilizando regras genéricas (\emph{wildcards}) para evitar instalar
separadamente fluxos é apresentado em \cite{richard2011openflow}. 
Esses fluxos reduzem o volume de pacotes enviados para o controlador.

Um balanceamento de carga dinâmico para \emph{cluster} de computadores em 
ambientes virtualizados utilizando OpenFlow é mostrado em 
\citep{chen2014design}.
Um algoritmo de balanceamento de carga chamado \emph{Server-based} (SBLB) 
é proposto. 
Dinamicamente ele recupera dados da carga nos servidores para a tomada de 
decisão do balanceamento de carga. 

A proposta do presente trabalho é uma extensão do módulo de balanceamento
de carga do controlador POX \citep{pox2015}.
São reduzidos a perda de pacotes e o atraso na resposta através da instalação
de fluxos reversos. 
São aumentados a disponibilidade e a largura de banda da rede.
Além disso, a distribuição de carga foi aplicada de maneira mais justa entre
os servidores de aplicação.


\section{Projeto de implementação}

A solução resume-se em um balanceador de carga TCP.
A implementação não lida com DNS (\emph{domain name service}).
O balanceador de carga toma decisões avaliando a carga dos servidores
(\emph{load avarage}), o número de pendências na fila de requisições do 
servidor HTTP e o número requisições já respondidas.
Os dados monitorados pelas políticas de balanceamento de carga foram coletados
através de um comutador \emph{OpenFlow} e através do sistema de arquivos 
distribuído NFS (\emph{Network File Syste}).

O controlador POX \citep{pox2015} foi adotado como base da solução proposta.
A arquitetura baseada em eventos permite que módulos atuem como produtores
e consumidores de eventos.
O módulo \emph{core} encapsula o controle de eventos.
Novos eventos podem ser registrados por outros módulos.

\subsection{Balanceador de carga}

O módulo \emph{load\_balancer} foi implementado como uma classe.
Esse módulo se registra no \emph{core} no momento em que o controlador é 
executado.
Para inicialiar o módulo é necessário informar uma lista de endereços IPs
que representam a lista de servidores a serem balanceados.
Um IP lógico deve ser informado ao módulo ao qual todas as requisições durante
os experimentos serão direcionadas.
Cada requisição direcionada ao IP lógico é direcionada a um dos servidores 
cuja carga, no momento, seja a menor.

\subsection{Fluxo de trabalho do controlador}

Novos pacotes na rede geram pacotes de entrada \emph{PacketIn} no controlador.
O balancedor de carga avalia o novo fluxo e instala uma regra no 
comutador baseado em uma política de balanceamento de carga.
Os demais pacotes similares a esse fluxo são encaminhados para o mesmo 
servidor.
Após algum tempo, os fluxos expiram no comutador, permitindo maior 
dinamismo na rede.

A decisão de qual servidor escolher para encaminhar os pacotes depende 
diretamente da política de balancemanto de carga. 
Através do sistema de arquivos remoto a carga dos servidores foi monitorada.
A tabela \ref{tab:monitor} apresenta os valores coletados de cada servidor.

\begin{table}
\begin{center}
\begin{tabular}{ |c|c|c| } 
 \hline
 Carga média & Fila de pendências & Número de \\ 
  de CPU & do servidor HTTP & requisições respondidas \\
 \hline
\end{tabular}
\end{center}
\label{tab:monitor}
\caption{Dados monitorados de cada servidor HTTP}
\end{table}


O fluxo da execução das requisições feitas ao serviço é mostrado na figura
\ref{fig:balancer-workflow}

\begin{figure}[htb!]
    \centering
    \includegraphics[width=\linewidth]{img/balancer-workflow}
    \caption{Fluxo de execução das requisições ao serviço de balanceamento 
    de carga}
    \label{fig:balancer-workflow}
\end{figure}

\textcolor{red}{Trocar essa imagem dos workflow de balanceamento de carga. 
Fazer no formato de diagrama de sequência.}

\subsection{}

Cinco políticas de balanceamento de carga foram utilizadas pelo módulo para 
distribuir a carga de trabalho entre os servidores.

\begin{enumerate}
    \item Round-robin
    \item Aleatória
    \item Carga
    \item Fila 
    \item Mistura
\end{enumerate}

\emph{Round Robin} é uma política que divide de maneira idêntica o volume de
requisições entre os servidores.

A política de balanceamento Aleatória escolhe a cada nova requisição, 
aleatoriamente, o servidor a responder pela requisição.

A política de balanceamento Carga é baseada em avaliar a carga de CPU 
de cada servidor e encaminhar a requisição para o servidor com a menor 
carga do momento.
    
Fila é uma política de balanceamento que avalia a quantidade de requisições
pendentes na fila de requisições do servidor HTTP.
    
A política de balanceamento Mistura combina as políticas Carga e Fila 
e escolhe o servidor com a menor valor da combinação.

\subsection{Servidor HTTP}

O servidor responsável por responder as requisições HTTP foi o 
\emph{Tornado} \citep{tornado}.
O \emph{Tornado} é um arcabouço e uma biblioteca HTTP assíncrona.
Nos experimentos, o \emph{Tornado} escreve, periodicamente, em um arquivo 
no sistema de arquivos os dados da sua fila de requisições pendentes,
a carga do servidor e o número de requsições já respondidas.
O balanceador de carga lê remotamente esses dados para compor os 
dados avaliados pelas políticas de balancemento de carga.

\subsection{Ambiente de simulação}

O ambiente de simulação é composto por 6 dispositivos.
Um controlador \emph{OpenFlow}, um \emph{switch OpenFlow}, um 
cliente e quatro servidores.
Cada computador está conectado diretamente a uma porta do \emph{switch}.

Os servidores são heterogêneos.
Cada computador executa um sistema operacional \emph{Linux} diferente.
Eles possuem arquiteturas, adaptadores de rede, mémórias primárias
e discos rígidos diferentes.
Essa característica é importante para avaliar o quão justo o balanceador 
de carga é um ambiente heterogêneo.

\section{Experimentos}

Essa seção apresenta os experimentos executados com requisições TCP e HTTP
dentro da rede.
Foram medidos largura de banda e latência.
Foram avaliados também, o tempo de resposta e a disponibilidade do serviço.

\subsection{Ambiente e testes}

Todas as requisições dos testes foram direcionadas ao IP lógico do 
serviço.
No cenário dos experimentos o 4 servidores possuíam IP do seguinte faixa de 
endereços: 192.168.1.100 até 192.168.1.103.
O endereço IP do serviço é 192.168.1.111.
Toda requisição para esse endereço IP deve ser balanceado.
O computador cliente possui o endereço IP 192.168.1.50.

O experimento TCP foi feito através do utilitário de linha de comandos 
\emph{iperf} através do computador cliente.
Cada conexão com 30 segundos de duração com medições feitas a cada 5 segundos.

O experimento com requisições HTTP possui duas abordagens.
Uma sequencial e outra paralela.
A abordagem sequencial utiliza a feramenta de linha de comandos \emph{curl}.
A em paralelo utiliza o utilitário \emph{httperf}.
Para cada caso de experimento, foram criadas 500 novas conexões.
Cada conexão com 50 requisições ao endereço IP do serviço.
No caso do experimento em paralelo, foram executadas 4 \emph{threads} para 
enviar as requisições.
Todas as requisiões foram feitas para a mesma URI (\emph{Uniform Resource
Identifier}).

\section{Proposta de solução}

\begin{frame}{Proposta de solução}

    \begin{figure}[h]
        \centering
        \includegraphics[scale=.8]{images/small-graph}
    \end{figure}
\end{frame}

\begin{frame}{Proposta de solução}
    
    \begin{itemize}
        \setlength{\itemsep}{.5cm}
        \item Uma abstração da rede em forma de um módulo em grafos 
        \item Esse grafo, em tempo real, é a base para computações na rede
        \item A solução objetiva simplificar a análise e o gerenciamento 
            em redes
    \end{itemize}

\end{frame}

\begin{frame}{Modelagem do grafo}

    \begin{itemize}
        \setlength{\itemsep}{.5cm}
        \item $G=(V, A)$, em que $V$ e $A$ são conjuntos finitos de vértices 
            e arestas, respectivamente.
        \item Cada vértice $v \in V$ é um computador ou \emph{switch}.
        \item Cada aresta $u \to v \in A$ é um enlace entre dois vértices.
        \item O peso das arestas $g(u, v)$ é quantidade de \emph{bytes} 
            trafegados na aresta entre os dois vértices.
    \end{itemize}
\end{frame}

\begin{frame}{Classes}

    \begin{columns}[T] % align columns
        \begin{column}{.33\textwidth}

            \begin{itemize}
                \setlength{\itemsep}{.5cm}
                \item{Graph}
                \item{GraphEntity}
                \item{Vertex}
                \item{Edges} 
                \item{GraphManager}
            \end{itemize}
        \end{column}%
        \hfill%
        \begin{column}{.67\textwidth}
            \begin{figure}[h]
                \centering
                \includegraphics[scale=.4]{images/graph-interfaces}
            \end{figure}
        \end{column}%
    \end{columns}

\end{frame}

\begin{frame}{Interface de programação}

    \begin{columns}[T] % align columns
        \begin{column}{.33\textwidth}

            \begin{itemize}
                \setlength{\itemsep}{.5cm}
                \item \emph{get\_vertex(id)}
                \item \emph{get\_adjacents(id)}
                \item \emph{snapshot()}
                \item \emph{get\_mst()}
            \end{itemize}
        \end{column}%
        \hfill%
        \begin{column}{.67\textwidth}
            \begin{figure}[h]
                \centering
                \includegraphics[scale=4]{images/gear}
            \end{figure}
        \end{column}%
    \end{columns}

\end{frame}

\begin{frame}{Integração entre módulos}

\begin{figure}[h!]
    \centering
    \includegraphics[scale=.55]{images/graph-module-integration}
\end{figure}

\end{frame}


\begin{table}[h!]
    \centering
    \begin{tabular}{ | l | l | l | l |}
    \hline
    \textbf{Experimento} & \textbf{Quantidade} \\ 
    \hline
    \hline Detecção de entidades & 1 \\ 
    \hline Remoção de entidades & 3 \\ 
    \hline Visualização em tempo real& 2 \\ 
    \hline Identificação de tráfego & 1 \\
    \hline Avaliação do controlador  & 8 \\
    \hline Avaliação do \emph{host\_tracker} & 10 \\
    \hline Avaliação da rede & 10 \\
    \hline
    \end{tabular}
    \caption{Tabela de experimentos}
    \label{tbl:experiments}
\end{table}

%\chapter{Aplicações}


\section{Árvore geradora mínima}

\section{Caminho mínimo}

\section{Balanceamento de carga}

%\chapter{Análise}


\section{Latência global}

\subsection{Rede convencional}
4 testes de iperf sem controlador
\subsection{Rede com a solução em grafos}
4 testes de iperf com controlador


\section{Latência do controlador}

\subsection{Novos fluxos}
Qual o atraso para novos fluxos.

\subsection{Instalação de fluxos}
Tempo de execução do controlador até instalar novos fluxos

\subsection{fluxos instalados}


\section{Largura de banda}

\subsection{Rede convencional}
\subsection{Rede com a solução em grafos}


\section{Roteamento}

\subsection{Roteamento com OSPF}
\subsection{Roteamento com \emph{reliable flood} através da solução em grafos}


\section{Sobrecarga de pacotes}

Reduzindo o intervalo de tempo com que os pacotes saem do controlador para 
sondagem de máquinas ativas e medindo o impacto na largura de banda à medida
que mais pacotes de sondagem trafegam na rede. Foram feitos 5 medições.


% Quantidade de pacotes a mais na rede por causa da solução

Contar total de packet In. Contar total de pacotes de sonda. Mostrar percentual
dos pacotes de sonda em relação ao total de packetIns.

\section{Carga (CPU) do controlador}

Pingall começando com 50 hosts simultâneos até 550. Avaliado o comportamento
da CPU no controlador.

\chapter{Trabalhos futuros}
\label{chap:future-work}

Como uma proposta futura, criar visualizador em tempo real do grafo que 
interaja com o administrador da rede e mostre, de uma maneira simples, 
toda a operação da rede.

Assim como o controlador POX, o módulo proposto, ao ter seu processo terminado,
não persiste as informações de estado da rede.
Todo o grafo computado é perdido.
Em função disso, um banco de dados em grafos, distribuído, poderia ser 
utilizado para persistir o grafo, e as informações do estado da rede, de 
maneira confiável e tolerante a falhas. 

Algoritmos genéricos em grafos poderiam ser implementados como uma biblioteca
para o módulo.
Assim, estabelecida uma periodicidade, esses algoritmos poderiam ser computados
no grafo e seus resultados publicados como extensão da API.

\section{Conclusão}


\begin{frame}{Trabalhos futuros}
    \begin{itemize}
        \item Avaliar se a redução de 2.5 MB na largura é constante em outros
            cenários
        \item Implementar um algoritmo de roteamento utilizando o grafo e
            comparar com OSPF e BGP
        \item Alterar mecanismo de sondagem para dobrar tempo quando o
            computador estiver ativo
        \item Disponibilizar uma biblioteca de algoritmos em grafos
    \end{itemize}
\end{frame}


\begin{frame}{Conclusão}
    \begin{itemize}
        \setlength{\itemsep}{.5cm}
        \item Grafos em SDN dinamicamente atualizado
        \item Baixo impacto na rede em se utilizar um módulo em grafos
        \item O controlador se manteve estável à medida que a rede crescia
    \end{itemize}
\end{frame}



\begin{frame}{Publicações}

    \begin{itemize}
        \item Aprovados
            \begin{itemize}
        \item \textbf{SBRC 2014} Análise e Gerenciamento de Rede através de 
            Grafos em Redes Definidas por Software
        \item \textbf{CNSM 2014} Network Management through Graphs in 
            Software Defined Networks
    \end{itemize}
    \end{itemize}

    \begin{itemize}
        \item Não publicados
    \begin{itemize}
        \item An efficient and dynamic Load Balancer based on 
            Software Defined Networking
    \end{itemize}
    \end{itemize}

    \begin{itemize}
        \item Em andamento
            \begin{itemize}
        \item Survey on graphs as a software defined network abstraction
        \item Grafos como uma primitiva do plano de controle para análise e 
            gerenciamento de Redes Definidas por Software
    \end{itemize}
    \end{itemize}
\end{frame}


\begin{frame}{Agradecimentos}

    \begin{columns}[T] % align columns
        \begin{column}{.33\textwidth}
            \begin{itemize}
                \item Colegas
                    \begin{itemize}
                        \item Frederico Martins
                        \item Bruno Pereira
                        \item Erik de Britto
                        \item Henrique Moura
                        \item Lucas Silva
                        \item Rodrigo Caetano
                    \end{itemize}
            \end{itemize}
        \end{column}%
        \hfill%
        \begin{column}{.33\textwidth}
            \begin{itemize}
                \item Professores
                    \begin{itemize}
                        \item Luiz Filipe
                        \item Dorgival Guedes
                        \item Marcos Augusto
                        \item Daniel Macedo
                        \item Antonio Loureiro
                    \end{itemize}
            \end{itemize}
        \end{column}%
        \hfill%
        \begin{column}{.33\textwidth}
            \begin{itemize}
                \item UFMG
                    \begin{itemize}
                        \item PPGCC
                        \item WiNet
                        \item DCC
                    \end{itemize}
            \end{itemize}
        \end{column}%
    \end{columns}
\end{frame}




% Bibliograph file
\ppgccbibliography{src/references}


%\input{apendice} % ARQUIVO CONTENDO OS APÊNDICES : OPCIONAL
%\input{anexo} % ARQUIVO CONTENDO OS ANEXOS: OPCIONAL

\end{document}
