\section{Openflow}



\subsection{Definição}

Em 2008 o protocolo OpenFlow foi publicado. Ele permitiu que pesquisadores
pudessem criar experimentos com novos protocolos em redes convencionais
\citep{nick2008openflow}.
O OpenFlow foi criado como um padrão aberto, o que permite que todos os 
fabricantes de equipamentos de redes possam habilitar seus produtos ao 
padrão.

O protocolo consiste em uma interface de programação para o switch. 
Assim, um programador pode, através de um programa, controlar a forma como 
um switch executa seu encaminhamento de pacotes. 
De uma maneira bem clara, o protocolo OpenFlow separa o plano de dados
do plano de controle, fazendo com que soluções SDN possam ser criadas 
e experimentadas.
Por ser uma solução de baixo custo, o OpenFlow obteve boa aceitação na 
academia e no mercado, dado o volume de empresas e pesquisas relacionadas
ou que utilizam o protocolo.

\subsection{Componentes}


\subsection{Arquitetura do Switch}

\subsection{Fluxos}

\subsection{Cabeçalho}

\subsection{Ações}

\subsection{Controlador}
