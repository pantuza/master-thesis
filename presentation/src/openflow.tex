
\subsection{O protocolo OpenFlow}

%
% Openflow
%
\begin{frame}\frametitle{Openflow}

    \begin{itemize}
    \item Se SDN é só um modelo, como implementá-lo?
    \end{itemize}
    	\begin{figure}[h]
        \centering
        \includegraphics[scale=0.5]{images/control-room.png}
    \end{figure}
\end{frame}



%
% Openflow
%
\begin{frame}\frametitle{Openflow}

    \begin{itemize}
    \item Openflow é um protocolo que possibilita experimentos e aplicações
          em SDN
    \end{itemize}
    	\begin{figure}[h]
        \centering
        \includegraphics[scale=0.3]{images/openflow.png}
    \end{figure}
\end{frame}


%
% Openflow
%
\begin{frame}\frametitle{Openflow}

    \begin{itemize}
    \item \href{http://archive.openflow.org/documents/openflow-wp-latest.pdf}{Artigo} publicado em 2008 
    \item Permitiu que pesquisadores pudessem criar experimentos com novos
          protocolos em redes convencionais 
    \end{itemize}

\end{frame}




%
% Openflow
%
\begin{frame}\frametitle{Porque é tão importante?}

    \begin{itemize}
    \item A arquitetura da Internet tem deficiências
    \item Inovações em rede custam caro
    \item A arquitetura continua acoplada à infraestrutura
    \item O Openflow define um padrão que qualquer fabricante de     
          \emph{hardware} de rede pode implementar
    \end{itemize}

\end{frame}
