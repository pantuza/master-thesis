%
% Introduction
%
\section{Introdução}


%
% Community emvolvment
%
\begin{frame}\frametitle{Apresentação}

	\begin{figure}[h]
        \centering
        \includegraphics[scale=0.25]{images/community.png}
    \end{figure}
\end{frame}


%
% Motivation
%
\begin{frame}\frametitle{Motivação}
   
    \begin{itemize}
        \setlength{\itemsep}{1cm}
        \item Aplicações em rede executam algoritmos em grafos para tomada de
            decisões ou para obter algum resultado.
        \item Em muitos casos, essa computação é feita em diferentes nós da 
            rede de maneira repetitiva.
        \item Em função da natureza logicamente centralizada do plano 
            de controle é possível minimizar a quantidade de aplicações 
            computando as mesmas informações. 
    \end{itemize}
\end{frame}


%
% Problem
%
\begin{frame}\frametitle{Problema}
    \begin{itemize}
        \setlength{\itemsep}{1cm}
        \item Uma visão topológica global é um dos principais aspectos do 
              paradigma das Redes definidas por software.
        \item Grafos são uma modelagem direta para representar de maneira 
              natural e precisa a topologia de uma rede.
        \item Em função disso, grafos deveriam ser um recurso básico, 
              uma premissa em controladores SDN para representar a rede.
    \end{itemize} 
\end{frame}


%
% Scientific contributions
%
\begin{frame}\frametitle{Contribuições científicas}
    \begin{itemize}
        \setlength{\itemsep}{1cm}
        \item O presente trabalho apresenta uma abstração da rede na forma 
              de um grafo para o gerenciamento de redes no plano controle 
              possibilitando automatizar detecção de falhas e 
              provisionamento para um grafo dinamicamente atualizado.
        \item Avaliações do controlador e da rede são apresentadas dos
              experimentos como prova de conceito.

    \end{itemize}
\end{frame}


%
% state of art
%
\begin{frame}\frametitle{Estado da arte}
    \begin{itemize}
        \item A ideia de visão topológica da rede está presente em vários 
            controladores SDN. O controlador NOX \citep{gude2008nox} 
            trabalha a topologia da rede baseando-se em eventos.
        \item Um banco de dados distribuído é proposto no controlador Onix 
            \citep{teemu2010onix}.
        \item Em \citep{hinrichs2009pratical} um sistema com regras de predição 
            estabelecem essa abstração.
        \item Linguagens de domínio específico (DSL) como o Frenetic 
            \citep{foster2011frenetic} e o Pyretic 
            \citep{monsanto2013composing} permitem a recuperação de 
            informações topológicas da rede.
        \item Em \citep{ramya2012dynamic} uma API em grafos é apresentado 
            dentro do context de computação na nuvem.
        \item Nossa proposta apresenta a avaliação da abordagem em grafos e 
            sua implementação no controlador POX \citep{pox2015}, de código 
            aberto e voltado para pesquisa.
    \end{itemize} 
 
\end{frame}
