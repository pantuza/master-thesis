%
% Introduction
%
\section{Introdução}


%
% Community emvolvment
%
\begin{frame}\frametitle{Apresentação}

	\begin{figure}[h]
        \centering
        \includegraphics[scale=0.5]{images/community.png}
    \end{figure}
\end{frame}


%
% Motivation
%
\begin{frame}\frametitle{Motivação}
   
    \begin{itemize}
        \setlength{\itemsep}{1cm}
        \item A Internet demanda que a infraestrutura evolua em paralelo com 
            as aplicações e serviços
        \item Algoritmos em grafos são base para diversas aplicações em rede
        \item Computação feita em diferentes nós da rede repetidamente
        \item Logicamente centralizado, o plano de controle permite 
            minimizar a quantidade de computações
    \end{itemize}
\end{frame}


%
% Problem
%
\begin{frame}\frametitle{Problema}
    \begin{itemize}
        \setlength{\itemsep}{1cm}
        \item Uma visão topológica global é um dos principais aspectos do 
              paradigma das Redes definidas por software.
        \item Grafos representam de maneira natural e precisa a topologia 
            de uma rede.
        \item Grafos deveriam ser um recurso básico, uma premissa em 
            controladores SDN
    \end{itemize} 
\end{frame}


%
% Scientific contributions
%
\begin{frame}\frametitle{Contribuições científicas}
    \begin{itemize}
        \setlength{\itemsep}{1cm}
        \item Uma abstração da rede na forma de um grafo dinamicamente 
            atualizado.
        \item Avaliações do controlador, da rede e do protocolo OpenFlow
        \item Avaliação de grafos como primitiva em SDN
    \end{itemize}
\end{frame}
