\subsection{Função da energia}

É possível adotar diferentes funções para estabelecer a energia
de cada pixel.
Aliás, várias funções de energia posem ser aplicadas em uma mesma imagem.
Note-se que pixels não destacados, 
que se misturam com os demais ao seu redor,
são menos ``importantes''.
A questão é: como decidir se um pixel é ou não destacado dos demais?

Considere-se cada pixel $(x,y)$ de uma imagem $\mathbf{I}$ como um número.
Suponha-se os conjuntos dos pontos dispostos 
nas linhas $\mathbf{I}_x$ e colunas $\mathbf{I}_y$
como uma amostragem de valores de uma função contínua.
Nesse caso, a derivada parcial em cada amostra de  $\mathbf{I}$,
$x \in \mathbf{I}_x$ e
$y \in \mathbf{I}_y$,
indica o ângulo da reta tangente ao ponto:
$\hat{x} = \frac{\partial}{\partial x}\mathbf{I}$ e
$\hat{y} = \frac{\partial}{\partial y}\mathbf{I}$.

\begin{equation}
\label{gradient}
\nabla f(x,y) = \frac{\partial f}{\partial x}\mathbf{I} + 
                \frac{\partial f}{\partial y}\mathbf{I}
\end{equation}

Estes ângulos definem um vetor de duas dimensões 
$\vec{v}(\hat{x},\hat{y})$ que ``aponta'' para a direção 
da mudança de cor em cada pixel da image.
A equação \ref{gradient} representa as variações de direção na
intensidade de cor, ou seja, o {\it gradianete} da imagem.
Quanto maior o valor do gradiente em um ponto, 
mais ``diferente'' ele é de seus visinhos (diferença cromática).
Como existem duas direções (dimensões) na imagem, 
a ``diferença'' total em um pixel $(x,y)$ pode ser obtida com a soma 
dos módulos dos ângulos em cada direção,
que é equivalente à $|\vec{v}|$.
Então, uma função de enrgia que busque destacar diferenças entre
pixels visinhos pode ser expressa na forma geral:

\begin{equation}
\label{energy}
e(\mathbf{I}) = 
    |\frac{\partial}{\partial x}\mathbf{I}| + 
    |\frac{\partial}{\partial y}\mathbf{I}|
\end{equation}

Duas questões permanecem: 
Como definir um número relevante para cada pixel; sua intensidade luminosa?
Como calcular as derivadas parcias em ambas as direções?

Os pixels são compostos por valores de intensidade 
das três cores básicas (RGB).
Para se definir um número único, 
é necessário considerar que o ser humano tem sensibilidade diferente 
para cada cor básica. 
Então, deve-se considerar a sensibilidade visual para 
as cores vermelha, verde e azul:

\begin{equation}
\label{luminosity}
    IL(R,G,B) = 0.30 \times R + 0.59 \times G + 0.11 \times B
\end{equation}

Aplicando-se esta função em todos os pixels, 
obtem-se as intensidades luminosas da imagem $\mathbf{I}$.
A imagem resultante é uma versão em ``escala de cinza''.

O operador de Sobel é uma aproximação imprecisa do gradiente da imagem,
mas é computacionalmente ``barato'' e simples, 
com bons resultados práticos. 
Aplicado em imagem, ele é um filtro que destaca as bordas dos objetos,
intensificando a diferença de um pixel com os demais.
Para cada pixel $(x,y)$, aplica-se uma média ponderada da luminosidade dos
pixels visinhos. 
Como é uma aproximação do gradiente, 
o filtro utiliza duas matrizes, uma para cada direção (horizontal e vertical).

Considere-se $A$ como uma matriz formada pela vizinhança de um pixel.
Sejam $M_x$ e $M_y$ as matrizes das médias ponderadas de cada direção.
Os gradientes aproximados são $G_x = M_x \times A$ e $G_y = M_y \times A$.
Para a média ponderada correta, o somatório dos pesos da matriz $M$ do
operador de Sobel deve ser zero.
Considerando a soma dos módulos dos gradientes:

\begin{equation}
\label{sobel}
    G = \sqrt{(G_x)^2 + (G_y)^2}
\end{equation}

A equação \ref{sobel} representa uma função de energia viável para a
técnica de redimensionamento por \emph{Seam Carving}\citep{shai2007seam}.

Para aplicar o operador de Sobel nos pixels 
das extremidades (bordas) da imagem, 
é necessário definir um valor ``default'' de seus visinhos inexistentes.
É possível considerar a intensidade luminosa como ``mais escura'' (zero).
Todavia, seguindo-se a orientação do trabalho,
utilizou-se a repetição das linhas próximas na forma similar a um ``espelho''.

Considere-se que a multiplicação das matrizes é da forma 
$G'_{(l,l)} = M'_{(l,m)} \times A_{(m,l)}$,
sendo $l$ o número de linhas e $m$ o número de colunas da matriz $M$.
Suponha-se o valor da posição $M_{(i,j)}$, que irá multiplicar 
pelo valor da posição $(j, i)$ da matriz $A$.
Partindo do pixel $(x, y)$, centro da matriz $A$, 
é necessário deslocar 
$\Delta{x} = x - \lfloor\frac{m}{2}\rfloor$ e 
$\Delta{y} = y - \lfloor\frac{l}{2}\rfloor$ 
para encontrar cada pixel correspondente à posição.
Em outras palavras, a posição $(j, i)$ correspondete na imagem $I_{(c_x, c_y)}$ é 
$c_x = \Delta{x} + i$ e $c_y = \Delta{y} + j$.
Perceba-se que, como se trata de multiplicação de matrizes, 
as dimensões da imagem em $A$ são inversas às dimensões de cada matriz $G$.


Considere-se uma imagem $I$ de dimensão máxima $(w, h)$, 
sendo $0 \le x < w$ e $0 \le y < h$.
Considere-se $n_x$ e $n_y$ como o número de colunas e linhas a serem espelhas:
$0 \le n_x \le \lfloor\frac{m}{2}\rfloor$ e 
$0 \le n_y \le \lfloor\frac{l}{2}\rfloor$.


\begin{itemize}
\item 
Para o espelhamento das colunas mais à esquerda 
($x \le \lfloor\frac{m}{2}\rfloor$) e linhas ao topo 
($y \le \lfloor\frac{l}{2}\rfloor$), 
basta-se observar que se 
$(c_x = \Delta{x} + i) < 0 $ e/ou 
$(c_y = \Delta{y} + j) < 0$.
Ou seja,
$c_x = -n_x$ e seu espelho deve ser $c_x = |n_x|$, assim como
$c_y = -n_y$ e seu espelho deve ser $c_y = |n_y|$:
em qualquer dos casos, mesmo sem espelhamento, a possição correspondente é 
$\mathbf{c_x = |\Delta{x} + i|}$ e/ou
$\mathbf{c_y = |\Delta{y} + j|}$.

\item
Nas colunas mais à direita 
($x \ge w - \lfloor\frac{m}{2}\rfloor$) e linhas ao fundo 
($y \ge h - \lfloor\frac{l}{2}\rfloor$), 
basta-se observar que 
$(c_x = \Delta{x} + i) \ge w$ e/ou 
$(c_y = \Delta{y} + j) \ge h$.
Como $x \le (w - 1)$ e $y \le (h - 1)$ (partindo do zero),
$c_x = (w - 1) + n_x$ e seu espelho deve ser $c_x = w - n_x$, assim como
$c_y = (h - 1) + n_y$ e seu espelho deve ser $c_y = h - n_y$.
Note-se que se $p = (q - 1) + r$ e $p' = q - r$, 
partindo de $p$ obtém-se 
$p' = 2q - \mathbf{p} - 1 = 2q - ((q - 1) + r) - 1 = 2q - q + 1 - r - 1 = q - r$.
Ou seja, é possível encontrar $p' = c_{\{x,y\}}$ 
sem se conhecer $r = n_{\{x,y\}}$.
Dai, se $c_x \ge w$, então $\mathbf{c_x = 2w - c_x - 1}$.
Analogamente, se $c_y \ge h$, então $\mathbf{c_y = 2h - c_y - 1}$.

\end{itemize}


